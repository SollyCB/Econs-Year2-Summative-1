\documentclass[12pt]{article}

\usepackage[english]{babel}
\usepackage[utf8]{inputenc}
\usepackage[a4paper, total={6in, 8in}]{geometry}
\usepackage{amsmath, mathtools , amssymb , amsthm }
\usepackage{graphicx}
\graphicspath{example/path} 
\raggedright

\author{Marking Code: Z0171688}
\title{Econs Theory Summative:\\ 
Explain how the economic growth of a developing nation may be different from that of a developed nation, and the challenges faced when attempting to increase its income per capita to the same level as that of developed nations.
}

\begin{document}

\maketitle
\pagebreak

\section{Introduction}
I will tackle this questions in two sections: the first will be a primarily theoretical 
analysis of the task, looking at models and growth, particularly the Solow Growth Model, 
in order to evaluate the stimuli and inhibitors of growth from the theoretical point 
of view. I will then go on to discuss historical and real world examples where these 
variables have manifested, for instance using Collier P. et al, 
\textit{Why Has Africa Grown Slowly?}

\section{The Solow Growth Model}
The Solow model can be used to analyse changes in the level of output of an economy 
overtime as the result of changes in physical capital (the stock of factories and 
machinery), human capital (how skilled and educated is the workforce), and technology 
(this can be considered somewhat of a catch-all for variables other than labor, or 
capital, for instance - unsurprisingly - closeness to the technological frontier). 
\newline

In a very basic form, the Solow model can be expressed as \( y = Af(k) \), 
where output per worker $y$ is a function of the capital per worker $k$ and technology 
$A$. The function $f()$ is assumed to give a diminishing rate of return (more specifically
it satisfies the Inada conditions). Plotting its graph looks like this:

\includegraphics[scale=0.85]{img/graph1.png}

Where \(f(k) = k^{1/3} \), and \( A = 0.5 \). 
\newline

When the output is low, a small change in the input causes a greater return on investment,
which would imply that poorer countries are able to grow more rapidly than richer ones in 
the short term, and that gradually their growth will converge with that of richer 
countries. However, this is only true if the circumstances under which two countries exist 
are equivalent. Continuing to examine closed economies, if one economy posses greater 
technology than another, say \( A = 0.7 \),  then naturally their 
output will be greater:

\includegraphics[scale=0.85]{img/graph2.png}
\newline

This is a very basic interpretation to demonstrate that economies which exist on an even 
playing field will converge, as they grow quickly at first, and gradually reach a steady 
state, but that they cannot converge if their circumstances differ. It also demonstrates 
that in order to achieve sustained growth, technological progress must happen: the graph 
demonstrates that increases in capital or labour gives diminishing returns, while 
technology is able to shift the level of output. The level of technology cannot remain 
constant however, as this does not give permanent growth. Furthermore, this example 
assumes a closed economy, which is not realistic. 
\newline

\subsection{Using Solow To Analyse Economies}

This section looks at some of the equations which underpin the Solow model, as they will be useful 
in analysing the question.
\newline

A better representation of the Solow model can be given as
\[ Y_{t} = K_{t}^{\alpha} \cdot (A_{t} \cdot L_{t})^{1-\alpha} \]
where $\alpha$ denotes the elasticity of output with respect to capital. $AL$ 
naturally shows \textbf{effective labour}: the workforce augmented by technology. This 
is very similar to the above (production as a function of capital $K$, technology $A$, and 
labour $L$, with technology as labour augmenting, \(Y = f(K, AL)\), and using the 
Cobb-Douglas for $f()$).
\newline

Assuming labour and the level of technology grow exogenously as 
\(L_{t} = L_{0} \cdot e^{nt} \) and \(A_{t} = A_{0} \cdot e^{gt} \), the effective 
workforce increases by \((n + g)\).
\newline 

Capital stock increases as investments are made. Imagining no trade
again, this investment is equivalent to the fraction of the output that is not 
consumed (savings), \(S_{t} = i_{t} = sY_{t}\). Capital stock will also decay over time 
by some amount $\delta$. Therefore, the change in capital stock with respect to time is 
\( \dot{K} = sY_{t} - \delta K_{t} \). 
\newline 

We can then write the production function in terms of $y$, the output per effective 
unit of labour. (This can be used to measure wealth creation, as it is an indirect 
indicator of per capita income.) 

\[y_{t} = \frac{Y_{t}}{A_{t}L_{t}} = \frac{K_{t}^{\alpha} \cdot (A_{t} \cdot L_{t})^{1-\alpha}}{A_{t} \cdot L_{t}} = \frac{K_{t}^{\alpha}}{A_{t} \cdot L{t}} = k_{t}^{\alpha}\]

We can now arrive at the core equation of the of model, which gives the capital stock per unit 
of effective labour.

\[ \dot{k} = sk_{t}^{\alpha} - (n + g + \delta) \cdot k_{t} \]

The first term represents the investment per unit of effective labour, \(sk_{t} = sy_{t}\), the 
product of the savings rate $s$ and the output per effective unit of labour $y$, while the second 
term represents the amount that must be invested in order to keep $k$ from diminishing. This 
equation is important because we can infer from it that $k_{t}$ will approach the steady state 
$k^{*}$, when \(sk_{t}^{\alpha} = (n + g + \delta) \cdot k_{t}\). At this point, capital deepening 
ceases. Therefore we can infer from the Solow model that an economy will reach a \textit{balanced 
growth path}, or constant growth rate. As output per worker only grows due to the rate of 
technological progress, $A$ can be considered the measure of productivity. 

Returning specifically to the question, in a global trade setting of open economies, money 
should flow from rich to poor countries, assuming productivity levels are even, as poorer countries 
with lower capital per worker offer a higher marginal product, as they will not have reached the 
same steady state of the more developed economies. Their growth will then eventually also 
stabilise. 
\newline

This is often not the case however, as investments in the real world are often shared between 
the most developed nations. The Solow model is insufficient to explain why developed countries 
will often offer a greater return on investment than poorer ones. Therefore, I will now shift my 
answer away from theoretical analysis, to look at more real world examples of how growth manifests, 
and when poorer nations are able to grow. 
\newline

\section{Why Do Some Poor Countries Grow, And Others Do Not?}

\pagebreak
\textbf{References:}

Collier P, and Gunning J. W.,(1999) ‘Why has Africa grown slowly?’, Journal of Economic Perspectives 13 (3) pp 3–22,

Fernández-Villaverde, J. and Jones, C.I., 2020. Macroeconomic outcomes and COVID-19: a progress report (No. w28004). National Bureau of Economic Research.

Bucci, A. and Guerrini, L., 2009. Transitional dynamics in the Solow-Swan growth model with AK technology and logistic population change. The BE Journal of Macroeconomics, 9(1).
\end{document}
